\documentclass[10pt,a4paper,oneside]{article}
\usepackage[utf8]{inputenc}
\usepackage[english]{babel}
\usepackage[T1]{fontenc}
\usepackage{amsmath}
\usepackage{amsfonts}
\usepackage{amssymb}
\usepackage{graphicx}
\usepackage{hyperref}
\usepackage[left=3cm,right=2cm,top=2cm,bottom=4cm]{geometry}
\usepackage{amsthm}
\usepackage{float}
\author{František Dostál}
\newtheorem{theorem}{Věta}
\newtheorem{lemma}{Lemma}[theorem]
\title{Killer Sudoku Solver}
\begin{document}
\maketitle
\section{Problem Description}
Killer Sudoku is a variant on classical Sudoku, where so called "cages" are added to the problem. The rules as described here are sourced form \href{https://www.csplib.org/Problems/prob057/}{CSPLibrary}.
The cages are usually continuous sets of sudoku cells, however since the rules are not particulary clear on this point, we will assume general set of cells to be a cage, the decision being left to the puzzle's author.\\
The numbers in cages have to differ, just like they have to differ in rows and classic sudoku brackets but they also have to add up to a pre-set sum.\\
The size of a cage should be smaller than classic sudoku bracket but again it's not enforced rule and we will grant agency in this matter to the author of the puzzle.\\
However each cell should belong only to one cage at a time.
We will assume that the input sudoku can be non-standard, meaning that the square bracket can have arbitrary size $ B\times B$ , therefore the whole puzzle being of size $ N\times N$ where $ N = B^{2} $. 

\section{Solution}
\section{Experimental Results}
\end{document}